\documentclass[a4paper]{scrartcl}

\usepackage[english]{babel}
\usepackage[a4paper]{geometry}
\usepackage{fullpage}
\usepackage{amsmath}
\usepackage{graphicx}
\usepackage{comment}
\usepackage{hyperref}
\usepackage{url}
\usepackage{proof}

\usepackage{listings}
\lstloadlanguages{Haskell}
\lstnewenvironment{code}
    {\lstset{}%
      \csname lst@SetFirstLabel\endcsname}
    {\csname lst@SaveFirstLabel\endcsname}
    \lstset{
      basicstyle=\small\ttfamily,
      language=haskell,
      flexiblecolumns=false,
      basewidth={0.5em,0.45em},
      literate={+}{{$+$}}1 {/}{{$/$}}1 {*}{{$*$}}1 {=}{{$=$}}1
               {>}{{$>$}}1 {<}{{$<$}}1 {\\}{{$\lambda$}}1
               {\\\\}{{\char`\\\char`\\}}1
               {->}{{$\rightarrow$}}2 {>=}{{$\geq$}}2 {<-}{{$\leftarrow$}}2
               {<=}{{$\leq$}}2 {=>}{{$\Rightarrow$}}2 
               {\ .}{{$\circ$}}2 {\ .\ }{{$\circ$}}2
               {>>}{{>>}}2 {>>=}{{>>=}}2
               {|}{{$\mid$}}1               
    }

\title{Control Flow Analysis of monomorphic Haskell}
\author{Stijn van Drongelen {\tiny(3117537)},
        Tom Tervoort {\tiny(3470784)},
        Gerben van Veenendaal {\tiny(3460692)}}
\date{\today}

\begin{document}

\maketitle

\section{Introduction}
\section{Goal}
% TODO

To implement CFA on a subset of Haskell. Since we use \texttt{haskell-src} to
parse, this is already a huge scope limiting factor. We further narrow it
down to mainly just functions and data types. In other words, we don't
support:

\begin{description}
\item[Recursive let-bindings]
    TODO
\item[Type signatures]
    Although type signatures shouldn't be too hard to support, we choose
    not to do this. The extra code required to connect a type signature to
    the corresponding binding is not worth it, given that the other restrictions
    on our subset make type inference decidable.
\item[Guards]
    Having to deal with guards makes the process of simplifying \texttt{case}
    constructs very hard. We choose not to support them.
\item[Type classes]
    Type classes require a lot of extra work in the type inference algorithm.
\item[Literals]
    Numeric literals (and with certain extensions, \texttt{String} as well)
    are polymorphic with a type class restriction. Since we don't support type
    classes, we can't support literals.
\item[Do-notation]
    The \texttt{do}-notation desugars to the use of operators from the \texttt{Monad}
    type class, and we can't support them.
\item[List comprehensions]
    List comprehensions can be seen as limited forms of the \texttt{do}-notation.
    We could support them, but since not much more interesting than other
    features of the language, we choose not to.
\item [Modules (and imports)]
    Most examples of control flow analysis fit in a single module. Although
    analysis beyond module boundaries might be interesting, we simply don't
    have the time to implement proper support for modules.
\item [Foreign function interface]
    Supporting the FFI is even harder than supporting imports. We simply don't
    do this.
\end{description}

Furthermore, we initially choose to perform a monomorphic, monovariant analysis.

\section{The language}
We consider a desugared form of Haskell 2010.

\begin{code}
data Term = LiteralTerm     {literal :: Literal}
          | VariableTerm    {varName :: Name}
          | ApplicationTerm {lhsTerm :: Term, rhsTerm :: Term}
          | AbstractionTerm {argName :: Name, bodyTerm :: Term}
          | LetInTerm       {letTerm :: (Name, Term), inTerm :: Term}
          | CaseTerm        {exprTerm :: Term, alts :: [(Pattern, Term)]}
          | ListTerm        {terms :: [Term]}
          | TupleTerm       {terms :: [Term]}
          | FixTerm         {fixedTerm :: Term}

data Pattern = Variable Name
             | Pattern {ctorName :: Name, ctorArgs :: [Name]}

type Name = String

data Literal = IntegerLit Integer
             | DoubleLit Rational
             | StringLit String
             | CharLit Char
\end{code}

% TODO type inference over this language

\section{The analysis}

% TODO leveraging type inference to perform CFA

\section{Implementation}
\section{Results}
\section{Discussion}
\begin{itemize}
\item
    We treat lists and tuples as special types. A different approach might
    have resulted in less redundant code.
\end{itemize}

\section{Conclusion}

\end{document}
